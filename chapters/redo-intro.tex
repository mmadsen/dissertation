From our vantage point in 2019, it seems as if the study of cultural transmission is a formalized, model-driven, highly interdisciplinary field, encompassing the ``hard science'' side of most of the traditional social sciences including psychology, economics, and anthropology.  The field is growing and maturing, creating bridges between the social sciences, cognitive science, and the behavioral side of evolutionary biology, leading to a deeper conception of ``cultural evolution'' than past attempts to import Darwinian ideas into the social sciences.  Anthropology can rightly claim a central place in introducing and elaborating the idea of the ``transmission'' of culture, and it is the only discipline among those listed here for which cultural transmission is---and has been, for more than a century---a \emph{central organizing concept} for the discipline \citep{lyman2008cultural}.  And within anthropology itself, archaeology has been central to the framing of cultural transmission as one of our discipline's central pillars both in terms of theory and by providing evidence for the spatio-temporal structure of the products of transmission.

And yet, despite our long history with the concept, and despite a plethora of recent works which study cultural transmission models, we struggle to use those models to explain the archaeological record in convincing, testable ways, except at the largest scales.  


Kandler and Shennan \citeyearpar{Kandler20150905} note that even the most studied case---ceramic type frequencies from the Merzbach \emph{Linearbandkeramik} in Germany---has produced conflicting results, with some studies consistent with neutral models, and others rejecting neturality in favor of anti-conformist transmission models.  Few data sets receive this much concentrated re-analysis, so it is difficult to judge how prevalent this kind of lack of replicability would be across models and data sets discipline-wide,  but there is growing recognition that significant equifinalities (REFS) between different transmission models when the explanatory target are sets of artifact class frequencies.  

But we know this about the state of the art in transmission modeling:  the failure to achieve clarity in results is not due to lack of methodological sophistication.  The methods used in the most recent studies by Kandler, Crema, and O'Dwyer having evolved well beyond simple tests of neutrality borrowed from population genetics in favor of non-equilibrium and generative models combined with approximate Bayesian computation for comparative model fitting \citep{kandler2018generative,kandler2019analysing}.  Such methods make it possible to move beyond simple models, allowing models to handle changing population size, mixtures of transmission ``modes,'' or change in social network structures.  Generative methods also make it possible to calculate temporal differences and other non-equilibrium statistics, which are directly applicable to the inherently diachronic nature of archaeological data \citep{kandler2013non}.  

However powerful the combination of generative modeling and approximate Bayesian analysis can be, equifinality between transmission models remains a significant issue, as Kandler and Crema note in a comprehensive recent review \citet{kandler2019analysing}.  

Some of the issues we face connecting models to data arise because of the questions we insist upon asking.  The shadow of Boyd and Richerson's \citeyearpar{BR1985} pioneering book is understandably quite long, since it (along with neutral theory from theoretical population genetics) gave us a quantitative framework within which to pose, and potentially answer, questions about cultural transmission.  But their pioneering work also focused specifically on the effect of different cognitive biases on trait frequencies, and had little or nothing to say about how different modes of transmission might yield patterning in other variables that social scientists could measure, especially at scales larger than individual populations.  Despite the fact that archaeology operates at very different scales than those Boyd and Richerson modeled, archaeologists interested in cultural transmission quickly adopted their detailed models, and attempts to identify ``biased'' transmission have grown to dominate the archaeological study of cultural transmission.  In part, this focus on models of transmission ``bias'' derives from the attempt to create methods to detect signatures for selection and selective forces in the history of artifact classes of technologies.  But this focus on 

l.....exclusive focus on distributional characteristics in a low-dimensional observable unit......


Specifically, we have been content for the most part to assume that the questions we should be asking about past populations concern their cognitive and learning biases.  Implicitly we have been assuming that models written in the language of social psychology should also directly speak the language of class frequencies in artifact assemblages.






Generative modeling with ABC fitting also allows ``observable'' statistics that are not synchronic or equilibrium measures.....

But Kandler and co-authors note that the methods do not remove 



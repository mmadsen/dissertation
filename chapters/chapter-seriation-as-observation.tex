\begin{description}[leftmargin=-1\labelwidth]
\item[\textsc{Overview}] \lipsum[1]
\item[\textsc{Contents}] \lipsum[2]
\end{description}




\section{Frequency Seriation}
\label{sec:seriation-frequency}



\section{Measuring Cultural Interaction Through Seriation}
\label{sec:measuring-interaction-seriation}



\section{Practical Techniques for Fully Solving the Seriation Problem}
\label{sec:full-solution-seriation-techniques}


\subsection{Lipo's Iterative Pairwise Seriation Method}
\label{sec:lipo-dissertation-seriations}

Seriation is a method of ordering empirical entities using analytic classes, given an ordering principle for those analytic classes \citep{dunnell1970seriation}.  The most common use of seriation is to derive chronological orderings, so the classes employed are stylistic in character (e.g., ``\culthistl types), and the ordering principles are heritable continuity of form and historical continuity.  In other words, homology due to \ct \citep{8961,lyman1997rise,o1999seriation,lyman2006measuring}.  

Ford's deterministic frequency seriation method has generally been non-quantitative, and was often done ``by eye.''  This offers no means of evaluating the statistical fit of a solution to the ordering principles.  Matrix methods, on the other hand, are inherently quantitative and could offer significance values for solutions, but collapsing the details of type frequencies into a single assemblage-level similarity value removes the possibility of determining the exactness of fit to unimodality for each type in a seriation.  Thus, a hybrid approach is required.  In his dissertation, \citet{Lipo2001b} elaborated a statistical approach to testing deterministic seriation orderings that grew out of work done with myself, R.C. Dunnell, and Tim Hunt \citeyearpar{Lipo1997}.  

The first step is to incorporate the effects of sampling and sample size upon type frequencies; unimodality of class frequencies in a seriation can only be evaluated within confidence limits governed by the size of samples.  Following \citet{Beals1945}, a confidence interval is calculated for each type frequency in each assemblage, using the normal approximation for a confidence interval on a binomial variable \citep{cochran2007sampling}.

Next, subsets of assemblages are found which seriate together (but cannot be further combined), within confidence limits.  This step was facilitated by the use of an Excel-based macro for constructing and manually manipulating assemblages into groups.\footnote{The software is open-source, available under a Creative Commons license at \url{http://lipolab.org/seriation.html}.  I am attempting to further automate this process, but if unsuccessful I will use the existing Excel solution in my dissertation research.}  Given a trial ordering of assemblages into groups which meet the distributional requirements for seriation, \citet{Lipo2001b} created a pairwise significance test to determine the likelihood that an entire ordering would occur simply due to chance sampling.  One approach would be to calculate pairwise Student's $t$ tests for type frequencies, but given the closed array of frequencies, repeated tests would rapidly lose statistical power, and there is no warrant to suppose that the sampling distribution of frequency differences in a seriation order are normally distributed.  In fact, if we consider that an assemblage is a time-averaged draw from the Ewens sampling distribution under the assumption of neutral transmission, we have every reason to believe that assemblage frequency differences are not normally distributed.  

\input{graphics/fig-iterative-pairwise-steps}

Thus, Lipo employed a Monte Carlo resampling approach to calculating a significance value.  The process is outlined with a single pair of assemblages in Figure \ref{fig:iterative-seriation}.  First, pairwise differences in frequencies are reduced to a series of directional comparisons (e.g., frequency of type 1 in assemblage A is less than the frequency of type 1 in assemblage B).  Then, using the frequencies of types as the resampling distribution, random assemblages of the same size as the original assemblage are generated, and their type frequencies calculated.  For each resampled assemblage, the directionality of type frequencies is tabulated.  If a resampled assemblage has the same directionality of type frequencies as the original, a ``match'' is scored (see subfigure (b) in Figure \ref{fig:iterative-seriation}).  If a resampled assemblage has different directionality for one or more types, that resample does not score a match.  The significance value for the pairwise ordering is thus the proportion of matches seen.  Lipo used 1000 resamples of each assemblage to perform pairwise comparisons, given that the method is computationally intensive (or was, in the late 1990's).  The end result is the ability to examine the goodness of fit of any trial seriation to the unimodal model expected from deterministic frequency seriation.  

This method constructs seriation solution groups which meet the ordering principles (and thus measure heritable continuity and the flow of cultural traits through a set of interacting populations over time), and carry information about the precision and significance level of the ordering.  Hereafter, when I use the term ``seriation,'' it will refer to this method, rather than the general universe of seriation techniques used by archaeologists.  

\subsection{Iterative Deterministic Seriation Solutions (IDSS)}
\label{sec:idss-seriation}


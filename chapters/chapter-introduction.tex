% \begin{description}[leftmargin=-1\labelwidth]
% \item[\textsc{Overview}] \lipsum[1]
% \end{description}


Because so much of human behavior is acquired within social groups and shaped by the manifold interactions between people that occur within those groups, there is a sense in which anthropology is the study of the \emph{results} of cultural transmission processes which create historical continuity for practices, norms, and institutions, as well as technologies and the rest of material culture.  Americanist archaeology, even more so than the rest of the discipline, came into its own by focusing upon the history of cultural expressions, treating their persistence and spatiotemporal distributions to be caused by a commonsensical and simplistic (but essentially correct) account of social learning and cultural transmission \citep{Lyman2008}.     

A lively subdiscipline within anthropology studies cultural transmission within the context of Darwinian evolutionary theory, treating it as a set of processes by which heritable cultural variation persists and spreads within human and primate populations.  This study is intensely interdisciplinary, especially in those areas where modern human populations are studied (where collaboration is often with social, cognitive, and evolutionary psychology), or non-human primates (where collaboration with zoologists, ethologists, and other biological sciences is critical).  



\section{The Place of Cultural Transmission Theory in Archaeology}
\label{sec:place-ct-in-archy}

Go through Lyman 2008 summary, importance to evolutionary archaeology but mention that one need not have a strict focus on Darwinian theory to find CT useful, one's ultimate research questions may lie in other areas, but the strong focus on spatiotemporal diffusion, adoption and innovation that cultural transmission theory brings are useful throughout the discipline.  





\section{What is Unique About Studying \CT in Archaeology?}
\label{sec:ct-archy-different}



\section{What Tools Do We Need To Study Past \CT?}
\label{sec:ct-archy-questions}



\section{Research Strategy}
\label{sec:research-strategy}



As previously noted, fully connecting the results of cultural transmission modeling to regional scale seriation algorithms has not been done in the published literature.  Multiple authors \citep[e.g.][]{Lipo1997,Neiman1990,Neiman1995,shepardson2006} have noted that the temporal traces of cultural transmission models often produce seriation-like curves of trait frequency, 

TODO:
Britton Shepardson -- prob need to move this stuff to seriation chapter, discuss unimodality and other criteria there.  here just mention the heuristic stuff like Neiman and Shepherson noting the gross similirity between traces of CT models and seriation curves.



\section{Summary}
\label{sec:plan-of-book}

The outline of the dissertation is as follows. Chapter \ref{chap:historical-perspective} lorem ipsum, sic dolor amet.  

\lipsum[3]






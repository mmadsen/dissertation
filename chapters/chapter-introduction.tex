% \begin{description}[leftmargin=-1\labelwidth]
% \item[\textsc{Overview}] \lipsum[1]
% \end{description}

From our vantage point in 2019, it seems as if the study of cultural transmission is a formalized, model-driven, highly interdisciplinary field, encompassing the ``hard science'' side of most of the traditional social sciences including psychology, economics, and anthropology.  The field is growing and maturing, creating bridges between the social sciences, cognitive science, and the behavioral side of evolutionary biology, leading to a deeper conception of ``cultural evolution'' than past attempts to import Darwinian ideas into the social sciences.  Anthropology can rightly claim a central place in introducing and elaborating the idea of the ``transmission'' of culture, and it is the only discipline among those listed here for which cultural transmission is---and has been, for more than a century---a \emph{central organizing concept} for the discipline \citep{lyman2008cultural}.  And within anthropology itself, archaeology has been central to the framing of cultural transmission as one of our discipline's central pillars both in terms of theory and by providing evidence for the spatio-temporal structure of the products of transmission.

And yet, despite our long history with the concept, and despite a plethora of recent works which provide formalized models and tools, we struggle to use our models to explain the archaeological record in convincing, testable ways.  Kandler and Shennan \citeyearpar{Kandler20150905} note that even the most studied case---ceramic type frequencies from the Merzbach \emph{Linearbandkeramik} in Germany---has produced conflicting results, with some studies consistent with neutral models, and others rejecting neturality in favor of anti-conformist transmission models.  Few data sets receive this much concentrated re-analysis, so it is difficult to judge how prevalent this kind of lack of replicability would be across models and data sets discipline-wide.  But we know this about the published work on the Merzbach:  the failure to achieve clarity in results is not due to lack of theoretical sophistication; the methods involved having evolved well beyond simple tests of neutrality borrowed from population genetics in favor of non-equilibrium and generative models combined with Bayesian analytical methods.  

The difficulties we face in using simple models of cultural transmission as explanations for archaeological data stem from the fact that our models are frequently \emph{equifinal} in their empirical predictions.  Equifinality can occur for 

The sources of, and remedies for, equifinalities between cultural transmission models, have been a major thread of research in the last decade.  My own work has been strongly dedicated to first identifying such equifinalities, and then changing our models and units of observation to eliminate them, if possible.  This dissertation presents five ``steps'' in my research program around equifinality in cultural transmission modeling.  

\section{Why Equifinality Plagues Our Application of Cultural Transmission Models in Archaeology}
\label{sec:equifinality}

Some of the issues we face connecting models to data arise because of the questions we insist upon asking.  The shadow of Boyd and Richerson's \citeyearpar{BR1985} pioneering book is understandably quite long, since it (along with neutral theory from theoretical population genetics) gave us a quantitative framework within which to pose, and potentially answer, questions about cultural transmission.  But their pioneering work also focused specifically on the effect of different cognitive biases on trait frequencies, and had little or nothing to say about how different modes of transmission might yield patterning in other variables that social scientists could measure, especially at scales larger than individual populations.  Despite the fact that archaeology operates at very different scales than those Boyd and Richerson modeled, archaeologists interested in cultural transmission quickly adopted their detailed models, and attempts to identify ``biased'' transmission have grown to dominate the archaeological study of cultural transmission.  Specifically, we have been content for the most part to assume that the questions we should be asking about past populations concern their cognitive and learning biases.  Implicitly we have been assuming that models written in the language of social psychology should also directly speak the language of class frequencies in artifact assemblages.  

There are several reasons why models about individual-level learning biases may not be appropriate subjects for archaeological investigation, even if we were interested in understanding the details of how social learning operated in past populations.  First, we need to understand how the formation processes of the archaeological record affect the nature of the assemblage frequencies we actually measure from fieldwork, and how these can be very different from the kinds of synchronic frequency counts our models appear to naturally predict.  The phenomena we seek to explain in archaeology are always aggregated in various ways; at a minimum the typical archaeological deposit is a time averaged record of many events of occupation and activity \citep{bailey2007time,bailey1981concepts,binford1981behavioral,8981,stein1987deposits}.  Following earlier work in zooarchaeology \citep{Lyman2003}, I examined how time averaging interacts with biased and unbiased transmission models and creates \textit{equifinality} between them which worsens with the amount of time over which assemblages accumulate \citep{}{Madsen2012TA}.  The first case study presented in this dissertation reflects my early work on this issue. 

Second, realistic populations always contain mixtures of cognitive biases, and this is rarely reflected in our modeling.  It is a real and open question whether we can identify biased social learning in observational data, outside of controlled laboratory experiments.  When formulating the core models of social learning processes, model simplicity is important.  Simple models allow us to analyze the behavior of the model over its range of parameters, and if models are simple enough, we may even be able to do this analytically and derive relationships between different models.  The history of theoretical population genetics demonstrates how critical it is to start simple and add complexity one step at a time.  But the history of population genetics also shows that the study of real populations also force us to go beyond our initial simple models, to study mixtures of processes which may no longer be analytically tractable.  Real populations are always mixtures of people with varying cognitive biases (and, of course, our deployment of various cognitive biases changes situationally, and our tendency to have specific biases changes over the course of one's lifetime). 

This leads to an important question:  if the data we possess about past populations mainly come in the form of frequency distributions across a relatively small number of artifact classes, to what extent can we expect to distinguish more realistic models?  To what extent are more realistic models identifiable within artifact trait frequency data, or do most mixtures of neutral and biased transmission models lead to equifinality?  The second case study in this dissertation begins the study of such issues, using machine learning methods to determine what the limits of our ability to discriminate among data generating processes may be given realistic mixtures and issues such as sample size.  

Given that equifinality between the types of generally studied transmission models seems to be frequent if not ubiquitous, what is to be done?  The simple answer is that we need to ask questions, and develop models, appropriate to the empirical record we study, not the data we \emph{wish} we had for the models that first inspired us to study formal models of cultural transmission.  Our study of cultural transmission in past populations needs to be inherently diachronic, rather than attempting to shoehorn synchronic predictions into comparisons with time averaged observations, and we may need to use more than just trait frequency distributions as our observational units.

\section{Modeling Cultural Transmission at Archaeological Scales}
\label{sec:modeling-archy-scales}





third cause of equifinality is in the choice of variables or observational units.  This leads to two main threads of research, designed to get away from the sole fixation on trait frequency distributions.  

\section{Observational Units for Archaeological Scale Transmission Models}
\label{sec:observational-units}






so long as we do not examine the observational units we use, and expand our toolkit for generating the data we need to test our models \citep{tostevin2019content}.  



My own recent work is aimed at evaluating the potential for ''coarse graining'' cultural transmission models to archaeological scales, which means first and foremost, forming a new set of ''observable'' units which are appropriate to the diachronic, aggregated, and time averaged empirical record we study.  This dissertation is a progress report on forming such observables and examining how to use them to study the kind of data we \emph{actually have}, not the data that would make our task easier---or in other words, the data we \emph{wish} we had. 








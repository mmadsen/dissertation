% \begin{description}[leftmargin=-1\labelwidth]
% \item[\textsc{Overview}] \lipsum[1]
% \end{description}

From our vantage point in 2019, it seems as if the study of cultural transmission is a  formalized, model-driven, highly interdisciplinary field, encompassing the ''hard science'' side of most of the traditional social sciences including psychology, economics, parts of cognitive science, sociology, anthropology, a strong representation among the behavioral aspects of evolutionary biology and zoology, and an increasing number of physicists who have turned their mathematical and modeling talents away from the realm of the very tiny to the realm of the relational and social.  Indeed, at this point it seems as if anthropology and especially the sub-discipline of archaeology are followers and consumers of cultural transmission theory, not leaders in its development and application.  And yet anthropology is the only discipline of those named in which cultural transmission is---and has been, for more than a century---a \emph{central organizing concept} for the discipline \citep{lyman2008cultural}.  Why are we perceived as followers, even sometimes by ourselves?  

Part of the answer lies in the qualitative and descriptive heritage of anthropology and archaeology; until recently we imported much of our quantitative theory from outside the discipline, using it to form narratives, and still are heavy borrowers.  But I will argue in this work that part of the answer lies in a scale distinction between much of the individual-level work on cultural transmission theory and the phenomena that anthropology has traditionally dealt with — and that archaeology still does, of necessity.  That scale distinction is holding us back, unless we address it.

The phenomena we seek to explain in archaeology are always aggregate in nature, always time averaged records of many events of occupation and activity.  Ours is an observational science, not an experimental or laboratory discipline, and we are not in full control of the time spans over which the deposits we collect and excavate, and the ''assemblages'' that result from those collections, possess.  Although there is a robust community within archaeology working on problems that involve cultural transmission models and their predictions, ours is not an empirical record which informs on transmission chains and individual-level cognitive biases.  Ours is an empirical record of past behavior and cultural evolution that requires coarser-grained theory and models which deal in predictions about aggregate measures of the flow of cultural information and traits over broad spans of geography and history.

On some level, we know this, which is why the archaeological literature on cultural transmission theory and its applications has turned, after a period of ''polemic and prototypes'', to attempts to address the equifinalities that result from aggregation and time averaging on our ability to fit the commonly discussed cultural transmission models to archaeological data.  This recent work addresses the dynamic and empirical sufficiency of cultural transmission models in archaeology \citep{Lewontin1974}.  My own early work in this field straddles the line between ''polemic and prototypes'' and addressing these equifinalities (e.g., \citep{Lipo1997}; REFS).    

Much of the recent work on addressing empirical sufficiency is quite valuable, and along the way I will point out those studies by others which are particularly important for workers in the field who are attempting to solve substantive issues and apply cultural transmission theory.  But it is my central contention that archaeology will continue to be a ''borrower'' of theory, plagued by issues of equifinality and empirical sufficiency, so long as we continue to attempt to use ''observables'' which are derived largely from synchronic models and individual-level theories of cultural transmission.  My own recent work is aimed at evaluating the potential for ''coarse graining'' cultural transmission models to archaeological scales, which means first and foremost, forming a new set of ''observable'' units which are appropriate to the diachronic, aggregated, and time averaged empirical record we study.  This dissertation is a progress report on forming such observables and examining how to use them to study the kind of data we \emph{actually have}, not the data that would make our task easier---or in other words, the data we \emph{wish} we had. 
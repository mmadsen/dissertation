% \begin{description}[leftmargin=-1\labelwidth]
% \item[\textsc{Overview}] \lipsum[1]
% \end{description}

From our vantage point in 2019, it seems as if the study of cultural transmission is a formalized, model-driven, highly interdisciplinary field, encompassing the ``hard science'' side of most of the traditional social sciences including psychology, economics, and anthropology.  The field is growing and maturing, creating bridges between the social sciences, cognitive science, and the behavioral side of evolutionary biology, leading to a deeper conception of ``cultural evolution'' than past attempts to import Darwinian ideas into the social sciences.  Anthropology can rightly claim a central place in introducing and elaborating the idea of the ``transmission'' of culture, and it is the only discipline among those listed here for which cultural transmission is---and has been, for more than a century---a \emph{central organizing concept} for the discipline \citep{lyman2008cultural}.  And within anthropology itself, archaeology has been central to the framing of cultural transmission as one of our discipline's central pillars both in terms of theory and by providing evidence for the spatio-temporal structure of the products of transmission.

And yet, despite our long history with the concept, and despite a plethora of recent works which provide formalized models and tools, we struggle to use our models to explain the archaeological record in convincing, testable ways.  Kandler and Shennan \citeyearpar{Kandler20150905} note that even the most studied case---ceramic type frequencies from the Merzbach \emph{Linearbandkeramik} in Germany---has produced conflicting results, with some studies consistent with neutral models, and others rejecting neturality in favor of anti-conformist transmission models.  This failure to achieve clarity is not due to lack of theoretical sophistication; the methods involved having evolved well beyond simple tests of neutrality borrowed from population genetics in favor of non-equilibrium and generative models combined with sophisticated Bayesian methods and large amounts of computational resources.  

Some of our issues connecting models to data arises because of our choice of questions and formal models.  The shadow of Boyd and Richerson's \citeyearpar{BR1985} pioneering book is understandably quite long, since it (along with neutral theory from theoretical population genetics) gave us a quantitative framework within which to pose, and potentially answer, questions about cultural transmission.  But their pioneering work also focuses specifically on the effect of different cognitive biases or ways of forming the set of models one might imitate, rather than larger scale issues of spatiotemporal patterning, for example.  Archaeologists interested in cultural transmission quickly adopted Boyd and Richerson's detailed models, and attempts to identify ``biased'' transmission have grown to dominate the archaeological literature on formal transmission modeling.  We have been content, for the most part, to assume that cultural transmission models written in the language of social psychology should also directly speak the language of class frequencies in artifact assemblages.  The inability Kandler and Shennan \citeyearpar{Kandler20150905} noted to clearly distinguish between biased and unbiased transmission in well studied cases such as the Merzbach should be a sign that social psychological models do not, in fact, have easy or direct implications for archaeological data.  

The difficulties we face in using simple models of cultural transmission as explanations for archaeological data sets stem from the fact that our models are \textit{equifinal} in their empirical predictions.  The sources of, and remedies for, equifinalities between cultural transmission models, have been a major thread of research in the last decade.  My own work has been strongly dedicated to first identifying such equifinalities, and then changing our models and units of observation to eliminate them, if possible.  This dissertation presents five ``steps'' in my research program around equifinality in cultural transmission modeling.  

First, we need to understand how the formation processes of the archaeological record affect the nature of the assemblage frequencies we actually measure from fieldwork, and how these can be very different from the kinds of synchronic frequency counts our models appear to naturally predict.  The phenomena we seek to explain in archaeology are always aggregated in various ways; at a minimum the typical archaeological deposit is a time averaged record of many events of occupation and activity \citep{bailey2007time,bailey1981concepts,binford1981behavioral,8981,stein1987deposits}.  Following earlier work in zooarchaeology by Lyman \citeyearpar{Lyman2003}, Luke Premo \citeyearpar{Premo2014} and myself \cite{Madsen2012TA} examined how time averaging interacts with biased and unbiased transmission models and creates \textit{equifinality} between models which worsens with the amount of time over which assemblages accumulate.  The first case study presented in this dissertation reflects my early work on this issue.

A second major source of equifinality is that our models might make distinguishable predictions given various ``pure'' states and parameter values, but make equifinal predictions when we construct more realistic population models.  We often make simplifying assumptions when we formulate models, often to gain analytic tractability.  Such simplifications have been critical to both population genetics and cultural transmission theory (which borrows most of its mathematical structure directly from theoretical popyulation genetics).  Without such simplifications, Neiman's \citeyearpar{Neiman1995} seminal contribution would not have been possible, giving rise to a rich lineage of work applying the implications of neutral theory to artifact class frequency distributions (including some of my own early work, e.g., \citealp{Lipo1997}).  But nobody really believes that real populations are composed of individuals who are all ``conformists'' or essentially unbiased in their adoption of traits, or any other ``mode'' we have collectively modeled in the cultural transmission literature.  Real populations are always mixtures of people with varying cognitive biases (and, of course, our deployment of various cognitive biases changes situationally, and our tendency to have specific biases changes over the course of one's lifetime).  The second case study in this dissertation begins the study of such issues, using machine learning methods to determine what the limits of our ability to discriminate among data generating processes may be given realistic mixtures and issues such as sample size.  


** what this means for choice of variables - since that's a key conclusion in the classifier paper

** third cause of equifinality is in the choice of variables or observational units.  This leads to two main threads of research, designed to get away from the sole fixation on trait frequency distributions.  

** seriation

** structured information models  




so long as we do not examine the observational units we use, and expand our toolkit for generating the data we need to test our models \citep{tostevin2019content}.  



My own recent work is aimed at evaluating the potential for ''coarse graining'' cultural transmission models to archaeological scales, which means first and foremost, forming a new set of ''observable'' units which are appropriate to the diachronic, aggregated, and time averaged empirical record we study.  This dissertation is a progress report on forming such observables and examining how to use them to study the kind of data we \emph{actually have}, not the data that would make our task easier---or in other words, the data we \emph{wish} we had. 








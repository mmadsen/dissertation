\begin{description}[leftmargin=-1\labelwidth]
\item[\textsc{Overview}] \lipsum[1]
\item[\textsc{Contents}] \lipsum[2]
\end{description}





\section{Constructing a Coarse Grained Theory}
\label{sec:rq-coarse-graining}

The overarching goal of this research is to improve the empirical sufficiency of \glslink{glos:microevolutionary}{microevolutionary} models of \ct as explanations for archaeological data.  Those data are usually (if not always) expressed as a count or relative frequencies of analytical classes, designed by the archaeologist.  Thus, the main theoretical questions here revolve around the dynamics and behavior of \ct models as observed through a \glslink{glos:designspace}{design space} given by a paradigmatic or hybrid classification.  

This task would be a large one if it were necessary to employ many of the detailed models of \ct formulated by anthropologists and psychologists to capture the details of bias and model selection between individuals.  However, since we know that populations are also hetergeneous with respect to the rules and preferences of individuals, a likely outcome is that transmission within such populations, especially when additionally averaged over long durations, will result in an ``averaged'' view of transmission which may be unbiased at the population level, even if individuals are highly biased in their individual interactions.  Thus, the first research question aims to address the question of whether the ``search space'' of transmission models can be reduced, for archaeological investigations over large regions of space and spans of time, to employ only \ac{UCT} as our theory of \ct.  

Given that \ac{UCT} turns out to be sufficient as a modeling framework for archaeological applications of \ct models, the next three research questions address the dynamics of \ac{UCT} as observed in classificatory design spaces, 




\subsection{Measuring \CT Given Time Averaged Observations}
\label{sec:rq-cg-timeaveraging}


\subsection{Measuring \CT Through Archaeological Classifications}
\label{sec:rq-cg-classification}


\subsection{Do Fine Grained \CT Models Converge to Neutral Theory?}
\label{sec:rq-cg-convergence}




\section{Applying Coarse Grained CT Theory}
\label{sec:rq-applications}



\subsection{Seriation and Regional-Scale CGCT Models}
\label{sec:rq-appl-seriation-behavior}



\subsection{Fitting CGCT Models To Lipo's LMV Ceramic Assemblages}
\label{sec:req-appl-lipo-lmv}










\begin{titlingpage}
\begin{center}

{\large University of Washington \par}

\vskip 0.75cm

 {\large \textbf{Abstract} \par}%
 
 \vskip 0.75cm
 
    { \thesisTitle\par}
    
    \vskip 0.5cm
  
      { \thesisAuthor\par}%
      
      \vskip 0.5cm
      
      { Chair of the Supervisory Committee:\\
      Research Associate Professor, James K. Feathers, co-Chair\\
	Associate Professor, Benjamin Marwick, co-Chair\\
	Anthropology\\}
  
  \vskip 0.75cm


\end{center}
	
%% Actual abstract goes here
\DoubleSpacing
\noindent Cultural transmission has long been a key organizing principle within anthropology, but the effort to formalize cultural transmission models and fit them to archaeological data is more recent, stimulated by work by Robert Dunnell in the 1970's.  Since then, the use of cultural transmission modeling in archaeology has branched into several research programs:  one macroevolutionary, employing phylogenetic methods; and one microevolutionary, employing models derived from population genetics.  A third research program, focused on intermediate or "mesoscopic" scales and seriation as a finer-grained counterpart to phylogenetic and cladistics, is being developed by Carl Lipo and the present author.  

This dissertation collects research papers by the author since 2012 which examine two questions.  First, are equifinality issues encountered in the microevolutionary research program solvable or do they prevent us from employing individual-scale models?  Second, to the extent that equifinality cannot be circumvented, can we construct better approaches at the mesosopic scale appropriate to coarse grained, time averaged data?

Two papers examine the first question, using simulation modeling and statistical methods to test whether theoretical models can be distinguished even in principle.  The first paper examines the effects of temporal aggregation, which is ubiquitous in the archaeological record, on our ability to distinguish between cultural transmission models, and finds significant issues in doing so with time averaged data.  The second paper examines the effects of population heterogeneity in social learning modes, which is well documented from living human and animal populations.  I find that heterogeneous mixtures of social learning rules can be identified statistically, but only with synchronic censusing of the population, and that time averaging and small samples render mixtures indistinguishable from pure unbiased copying.  

Turning to the second question, three papers continue my long-term research into reshaping the classical seriation method into a tool for tracing the structure of cultural transmision at regional scales.  One short paper examines the combinatorial structure of the seriation problem when we admit multiple subsolutions.  A second paper seeks to increase the size of possible seriations, which is necessary to incorporate significant spatial variation and yield a tool usable for investigating the history of cultural transmission in a region.  We increase the potential size of seriation solutions by switching from unimodality to distance minimization as the ordering criterion, yielding ``continuity'' seriation as a distinct method.  A third paper in this group then applies continuity seriation graphs as the observable variable, in a methodological study of how to construct models of how cultural transmission was structured at the regional scale.  This paper introduces ``interval temporal networks'' as a way to formalize our hypotheses about regional interaction and transmission, and explores a statistical method for summarizing the topology of seriation graphs, to assess their fit to our regional interaction models.

A final paper examines a different kind of mesoscale question:  how do we begin to model not just the spatiotemporal structure of past cultural transmission, but its \emph{content} as well.  The chapter models the dependency structure of the knowledge required to construct complex artifact types, through the ``prerequisites'' needed for each step, and introduces a model where transmission of subsequent traits requires learning their prerequisites first.  This simplified model of ``structured'' cultural traits is then used to explore the ``learning hypothesis'' for behavioral modernity, by looking at the richness and depth of knowledge gained when transmission is mostly accomplished by simple imitation compared to learning via a teacher.  The results are suggestive that the learning hypothesis can account for the increased richness of ``behaviorally modern'' hominids, and more importantly, points the way to more substantive and technologically informed cultural transmission models.  


\end{titlingpage}